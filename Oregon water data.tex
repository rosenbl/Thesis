
\documentclass[12pt]{amsart}
\usepackage{geometry} % see geometry.pdf on how to lay out the page. There's lots.
\geometry{a4paper} % or letter or a5paper or ... etc
% \geometry{landscape} % rotated page geometry

% See the ``Article customise'' template for come common customisations

\title{Oregon water data}
\author{Blake Rosenthal}
\date{} % delete this line to display the current date

%%% BEGIN DOCUMENT
\begin{document}

\maketitle

\section{Background}
A wide variety of data can be used for kriging, provided it meets a few necessary conditions. Most importantly, the data must be a sample from a spatially continuous random process. Since kriging provides a point prediction for any location within a region, a discrete or discontinuous random process cannot be used. \\

This thesis will analyze Oregon ground water depth using data from Oregon's Water Resources Department. The data itself is a collection of logs recorded by Oregon-bonded well drillers and includes such information as the drilling date, the depth of the well, the depth of the first occurrence of water, and flow rate. The Water Resources Department uses this data to monitor water quality throughout the state of Oregon. For the purposes of this thesis, the data represents a partial sampling from a mostly continuous supply of subterranean water. By applying kriging to the available sample, it is possible to make predictions for unsampled locations in Oregon. \\

Oregon's records contain nearly 500,000 wells in the state. Since the individual contractors are responsible for recording their own observations, much of the data is incomplete. Only a handful of the observations include the latitude and longitude coordinates necessary for spatial prediction. Of this subsample, a five-year date window was selected for the sake of accuracy. \\

\end{document}